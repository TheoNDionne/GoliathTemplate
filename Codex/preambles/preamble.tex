%------------------Math Setup-------------------

%%% Packages

\usepackage{amsmath}        % math environments
\usepackage{mathtools}      % tools for math formating
\usepackage[nice]{nicefrac} % nicer fractions
\usepackage{cancel}         % allows to scratch expressions.
\usepackage{slashed}        % allows to slash individual characters.
\usepackage{xargs}          % better handling of optional arguments for commands
\usepackage{braket}         % convenient Dirac notation

%%% Custom macros

\usepackage{macros} % custom latex macros (macros.sty)

%------------Fonts (Body and math)--------------

%%% Main font (EB Garamond)

\usepackage{fontspec} % Finer font selection (requires Lua/XeLaTeX)
\setmainfont{EB Garamond}[
  Ligatures   = TeX, % ligatures
  OpticalSize = On, % optical weight adjustment
  % Numbers     = OldStyle, % can be a little hard to read
  SmallCapsFeatures = {LetterSpace=5} % Adds spaces for small caps
]

%%% Math font (Libertinus Math)

\usepackage{unicode-math} % Finer math font selection
\setmathfont{Libertinus Math}[ % Nice neutral universal math font
    Scale=MatchLowercase
]
\setmathfont[ % Adding back a familiar mathcal
    range = {\mathcal, \dagger, \ddagger},
    Scale = MatchLowercase
]{Latin Modern Math}

%-----------------Misc packages-----------------

\usepackage[french]{babel} % language support

%%% French specific names
\addto\captionsfrench{\renewcommand\frenchtablename{{\FBfigtabshape Tableau}}} % rename table name (french)

% Names for theorem box type environments
\addto\captionsfrench{\def\theoremname{Théorème}}
\addto\captionsenglish{\def\theoremname{Theorem}}

\addto\captionsfrench{\def\definitionname{Définition}}
\addto\captionsenglish{\def\definitionname{Definition}}

\addto\captionsfrench{\def\corollaryname{Corollaire}}
\addto\captionsenglish{\def\corollaryname{Corollary}}

\addto\captionsfrench{\def\examplename{Exemple}}
\addto\captionsenglish{\def\examplename{Example}}

\addto\captionsfrench{\def\lemmaname{Lemme}}
\addto\captionsenglish{\def\lemmaname{Lemma}}


\usepackage{setspace} % line spacing
\usepackage{cite} % in text citations
\usepackage{microtype} % small typography tweaks
\usepackage{graphicx}		% figure insertion
\usepackage{subfig}			% allows for subfigures
\usepackage{geometry}		% document geometry
\usepackage{fancybox}		% boxes, frames etc
\usepackage{url}            % typographically sound URLs
\usepackage{float} % Fixes hacky geometry problems
\usepackage[
format    = hang,
margin    = 5mm,
font      = small,
labelfont = bf,
labelsep  = space
]{caption} % tweak caption layout and format
\usepackage{csquotes} % For quotes
\newcommand\itsym{$\bullet$} % symbole pour les listes

%--------------------Tables---------------------

\usepackage{array} % tabular functions
\newcolumntype{C}{>{$\displaystyle}c<{$}} % centered math column
\newcolumntype{L}{>{$\displaystyle}l<{$}} % left aligned math column
\newcolumntype{R}{>{$\displaystyle}r<{$}} % right aligned math column
\renewcommand{\arraystretch}{1.5} % vertical spacing of tables
\usepackage{dcolumn} % allows for aligning of values wrt to the decimal

%---------------General layout------------------

\usepackage{theme/layout}

%-------------Colors and Hyperref---------------

% Pass hyperref options before loading pdfx
\PassOptionsToPackage{
    backref=page,
    pagebackref=true,
    hyperindex=true,
    bookmarks=true,
    pdfa
}{hyperref}

\usepackage[a-2b,mathxmp]{pdfx}[2018/12/22] % ???

% options PDF
\hypersetup{
    colorlinks=true,         % colorise les liens
    breaklinks=true,         % permet le retour la ligne dans les liens trop longs
    urlcolor=URLColor,       % couleur des hyperliens (doit inclure x11names dans xcolor ci-dessus)
    linkcolor=LinkColor,     % couleur des liens internes
    citecolor=CiteColor,     % couleur des liens de citation
    bookmarksopen=true,      % ouvre les signets PDF au départ
}

%%% Choose color theme

\usepackage{colorthemes/colors_marine} % choose a style

%%% Colored boxes

\usepackage{empheq}
\setlength{\fboxrule}{1pt} % sets the border width globally (or locally in a group)
\setlength{\fboxsep}{4pt}  % sets padding between frame and content

% Robust color boxes that can be used even within aligns
\makeatletter
\newcommand{\cboxed}[1]{%
  \mathchoice
    {\fcolorbox{MainColor}{MainColor!5}{$\displaystyle #1$}}
    {\fcolorbox{MainColor}{MainColor!5}{$\textstyle #1$}}
    {\fcolorbox{MainColor}{MainColor!5}{$\scriptstyle #1$}}
    {\fcolorbox{MainColor}{MainColor!5}{$\scriptscriptstyle #1$}}%
}
\makeatother

%------------------------------------------------BOXES
\usepackage[most]{tcolorbox}

%-------------------------------
% Define a reusable box style
%-------------------------------
\tcbset{
mytheoremstyle/.style={
    fonttitle=\bfseries\upshape, % bold upright title
    fontupper=\upshape, % upright body font
    arc=0mm, % sharpish corner
    colback=MainColor!5, % color of the background
    colframe=MainColor, % color of the frame
    width=\linewidth, % width of the box
    boxrule=0.5mm,
    top=1mm,
    bottom=1mm,
    left=2mm,
    right=2mm,
    enhanced
}
}

%-------------------------------
% Define all your theorem-like boxes using this style
%-------------------------------
\newtcbtheorem[auto counter,number within=chapter]{theorembox}%
    {\theoremname}{mytheoremstyle}{theorem}

\newtcbtheorem[auto counter,number within=chapter]{definitionbox}%
    {\definitionname}{mytheoremstyle}{definition}

\newtcbtheorem[auto counter,number within=chapter]{corollarybox}%
    {\corollaryname}{mytheoremstyle}{corollary}

\newtcbtheorem[auto counter,number within=chapter]{examplebox}%
    {\examplename}{mytheoremstyle}{example}

\newtcbtheorem[auto counter,number within=chapter]{lemmabox}%
    {\lemmaname}{mytheoremstyle}{lemma}

%-------------------------------------------------------------------------------
% Figures TiKz !!!TODO FIX THIS!!!
\usepackage{tikz}
\usetikzlibrary{
	calc,
	patterns,
	positioning,
    external,
    shapes,
    fit,
    backgrounds,
    arrows.meta,
	positioning,
    decorations,
    decorations.pathmorphing,
    decorations.markings,
    shapes.geometric,
    arrows
}
% \tikzexternalize[prefix=figures/pdf/]

\usepackage{pgfplots}
\pgfplotsset{compat=1.16}
\pgfdeclarelayer{background}
\pgfdeclarelayer{foreground}
\pgfsetlayers{background,main,foreground}

\usepackage{theme/tikzstyles}
%%%%%%%%%%%%%%%%%%%%%%%%%%%%%%%%%%%%%%%%%%%%%%%%%%%%%%%%%%%%%%%%%%%%%%%%%%%%%%%%
