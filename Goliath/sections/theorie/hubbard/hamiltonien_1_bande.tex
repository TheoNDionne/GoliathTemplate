

Dans sa forme la plus générale, le hamiltonien du modèle prend la forme suivante dans le contexte de la seconde quantification:
\begin{align}
    \mathrm{H} = \sum_{ij,\sigma} t_{ij}\mathrm{c}_{i\sigma}^\dagger\mathrm{c}_{j\sigma}+U\sum_i\mathrm{n}_{i\uparrow}\mathrm{n}_{i\downarrow}.\label{eq:hamiltonien_hubbard_1_bande}
\end{align}
Les opérateurs $\mathrm c^{(\dagger)}_{i\sigma}$ sont les opérateurs annihilation (création) fermioniques d'un électron sur le site $i$ de spin $\sigma$.
Ces derniers peuvent être exprimés dans la base réciproque via les transformée de Fourier:
\begin{align}
    \begin{split}
        \mathrm{c}^\dag_{i\sigma} = \frac{1}{\sqrt{N}}\sum_\ve{k}\Exp{-i\ve{k}\cdot\ve{r}_i}\mathrm{c}^\dag_{\ve{k}\sigma}
    \end{split}
    \begin{split}
        \mathrm{c}_{i\sigma} = \frac{1}{\sqrt{N}}\sum_\ve{k}\Exp{i\ve{k}\cdot\ve{r}_i}\mathrm{c}_{\ve{k}\sigma}
    \end{split}\label{eq:op_creat_anni_fourier}
\end{align}

\begin{figure}
    \centering
    \scalebox{2.5}{
    % \tikzset{external/force remake}
\tikzsetnextfilename{hubbard_1_band}
\begin{tikzpicture}

    % GRID
    \draw[step=1cm,lightgray,very thin, dashed] (-0.2,-0.2) grid (2.2,2.2);

    \foreach \x in {0,1,2}{
        \foreach \y in {0,1,2}{
            \node[circle,draw=MainColor, fill=SupportColor,inner sep=1pt,minimum size=7pt,] at (\x,\y) {};
        }
    }


    % SPINS
    \begin{pgfonlayer}{foreground}
        \draw[-to, semithick,rotate=90] (-.1,0)--(.1,0) ;

        \draw[-to, semithick,rotate=90] (0.9,-1.06)--(1.1,-1.06) ;
        \draw[-to, semithick,rotate=90] (1.1,-.94)--(0.9,-.94) ;
        \node[right] at (1.05,1.2) {\footnotesize $U$};

        \node[right] at (0,-0.1) {\tiny $j$};
        \node[right] at (1,-0.1) {\tiny $i$};

        \draw[-to, semithick,rotate=90] (1.1,-2)--(0.9,-2) ;

        \draw[-to, semithick,rotate=90] (1.9,-1)--(2.1,-1) ;

        \node[above] at (0.5,0.15) {\footnotesize $t$};
        \draw[-to,black] ($(-.28,0)+(.4,.08)$) arc
        [
            start angle=130,
            end angle=50,
            x radius=0.6cm,
            y radius =0.4cm
        ] ;

    \end{pgfonlayer}

    \end{tikzpicture}
    }
    \caption[Représentation schématique du modèle de Hubbard à une bande.]{Représentation schématique du modèle de Hubbard à une bande. Les cercles sont les sites du réseau et les flèches sont des abstraction des électrons occupant ces même sites. Elles pointent dans la direction du spin de ces électrons. Un bond d'amplitude $t$ entre deux site ainsi qu'un double occupation de coût $U$ sont aussi explicitées.}
    \label{fig:hubbard_1_band}
\end{figure}
