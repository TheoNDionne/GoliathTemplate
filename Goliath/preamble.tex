%------------------Math Setup-------------------

%%% Packages

\usepackage{amsmath}        % math environments
\usepackage{mathtools}      % tools for math formating
\usepackage[nice]{nicefrac} % nicer fractions
\usepackage{cancel}         % allows to scratch expressions.
\usepackage{slashed}        % allows to slash individual characters.
\usepackage{xargs}          % better handling of optional arguments for commands
\usepackage{braket}         % convenient Dirac notation

%%% Custom macros

\usepackage{macros} % custom latex macros (macros.sty)

%------------Fonts (Body and math)--------------

%%% Main font (EB Garamond)

\usepackage{fontspec} % Finer font selection (requires Lua/XeLaTeX)
\setmainfont{EB Garamond}[
  Ligatures   = TeX, % ligatures
  OpticalSize = On, % optical weight adjustment
  % Numbers     = OldStyle, % can be a little hard to read
  SmallCapsFeatures = {LetterSpace=5} % Adds spaces for small caps
]

%%% Math font (Libertinus Math)

\usepackage{unicode-math} % Finer math font selection
\setmathfont{Libertinus Math}[ % Nice neutral universal math font
Scale=MatchLowercase
]
\setmathfont[ % Adding back a familiar mathcal
range = \mathcal,
Scale = MatchLowercase
]{Latin Modern Math}

%-----------------Misc packages-----------------

\usepackage[french]{babel} % language support
\usepackage{setspace} % line spacing
\usepackage{cite} % in text citations
\usepackage{microtype} % small typography tweaks
\usepackage{graphicx}		% figure insertion
\usepackage{subfig}			% allows for subfigures
\usepackage{geometry}		% document geometry
\usepackage{fancybox}		% boxes, frames etc
\usepackage{url}            % typographically sound URLs
\usepackage{float} % Fixes hacky geometry problems
\usepackage[
format    = hang,
margin    = 5mm,
font      = small,
labelfont = bf,
labelsep  = space
]{caption} % tweak caption layout and format
\newcommand\itsym{$\bullet$} % symbole pour les listes

%-------------------------------------------------------------------------------
% Pass hyperref options before loading pdfx
\PassOptionsToPackage{
    backref=page,
    pagebackref=true,
    hyperindex=true,
    bookmarks=true,
    pdfa
}{hyperref}

\usepackage[a-2b,mathxmp]{pdfx}[2018/12/22]

% \usepackage{theme/colors_vert_et_or}
\usepackage{theme/colors_crimson}

% options PDF
\hypersetup{
    colorlinks=true,         % colorise les liens
    breaklinks=true,         % permet le retour la ligne dans les liens trop longs
    urlcolor=URLColor,   % couleur des hyperliens (doit inclure x11names dans xcolor ci-dessus)
    linkcolor=LinkColor,  % couleur des liens internes
    citecolor=CiteColor,  % couleur des liens de citation
    bookmarksopen=true,      % ouvre les signets PDF au départ
}


%-----------------------------------------------------------------------------
% Tableaux
\usepackage{array} % généralise certaines fonctions de tabulation
\newcolumntype{C}{>{$\displaystyle}c<{$}} % colonne mathématique centrée
\newcolumntype{L}{>{$\displaystyle}l<{$}} % colonne mathématique alignée à gauche
\newcolumntype{R}{>{$\displaystyle}r<{$}} % colonne mathématique alignée à droite
\renewcommand{\arraystretch}{1.5}
\usepackage{dcolumn} % permet l'alignement sur le point décimal
\addto\captionsfrench{\renewcommand\frenchtablename{{\FBfigtabshape Tableau}}}

%---------------General layout------------------

\usepackage{theme/layout}

%-------------------------------------------------------------------------------
% Figures TiKz !!!TODO FIX THIS!!!
\usepackage{tikz}
\usetikzlibrary{
	calc,
	patterns,
	positioning,
    external,
    shapes,
    fit,
    backgrounds,
    arrows.meta,
	positioning,
    decorations,
    decorations.pathmorphing,
    decorations.markings,
    shapes.geometric,
    arrows
}
\tikzexternalize[prefix=figures/pdf/]

\usepackage{pgfplots}
\pgfplotsset{compat=1.16}
\pgfdeclarelayer{background}
\pgfdeclarelayer{foreground}
\pgfsetlayers{background,main,foreground}

\usepackage{theme/tikzstyles}
%%%%%%%%%%%%%%%%%%%%%%%%%%%%%%%%%%%%%%%%%%%%%%%%%%%%%%%%%%%%%%%%%%%%%%%%%%%%%%%%
